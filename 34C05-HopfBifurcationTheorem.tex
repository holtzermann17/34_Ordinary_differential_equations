\documentclass[12pt]{article}
\usepackage{pmmeta}
\pmcanonicalname{HopfBifurcationTheorem}
\pmcreated{2013-03-22 13:18:45}
\pmmodified{2013-03-22 13:18:45}
\pmowner{Daume}{40}
\pmmodifier{Daume}{40}
\pmtitle{Hopf bifurcation theorem}
\pmrecord{8}{33819}
\pmprivacy{1}
\pmauthor{Daume}{40}
\pmtype{Theorem}
\pmcomment{trigger rebuild}
\pmclassification{msc}{34C05}
\pmsynonym{Poincar\'e-Andronov-Hopf}{HopfBifurcationTheorem}

\endmetadata

% this is the default PlanetMath preamble.  as your knowledge
% of TeX increases, you will probably want to edit this, but
% it should be fine as is for beginners.

% almost certainly you want these
\usepackage{amssymb}
\usepackage{amsmath}
\usepackage{amsfonts}

% used for TeXing text within eps files
%\usepackage{psfrag}
% need this for including graphics (\includegraphics)
%\usepackage{graphicx}
% for neatly defining theorems and propositions
%\usepackage{amsthm}
% making logically defined graphics
%%%\usepackage{xypic}

% there are many more packages, add them here as you need them

% define commands here
\begin{document}
\PMlinkescapeword{planar}

Consider a planar system of ordinary differential equations, written in such a form as to make explicit the dependence on a parameter $\mu$:
\begin{eqnarray*}
x'&=&f_1(x,y,\mu) \\
y'&=&f_2(x,y,\mu)
\end{eqnarray*}
Assume that this system has the origin as an equilibrium for all $\mu$. Suppose that the linearization $Df$ at zero has the two purely imaginary eigenvalues $\lambda_1(\mu)$ and $\lambda_2(\mu)$ when $\mu=\mu_c$. If the real part of the eigenvalues verify
\[
\frac{d}{d\mu}\left(\Re\left(\lambda_{1,2}(\mu)\right)\right)_{|\mu=\mu_c}>0
\]
and the origin is asymptotically stable at $\mu=\mu_c$, then
\begin[roman]{enumerate}
\item $\mu_c$ is a bifurcation point;
\item for some $\mu_1\in\bar{\mathbb{R}}$ such that $\mu_1<\mu<\mu_c$, the origin is a stable focus;
\item for some $\mu_2\in\bar{\mathbb{R}}$ such that $\mu_c<\mu<\mu_2$, the origin is unstable, surrounded by a stable limit cycle whose size increases with $\mu$.

This is a simplified version of the theorem, corresponding to a supercritical Hopf bifurcation.
\end{enumerate}

Sometimes the Hopf theorem is called \emph{Poincar\'e-Andronov-Hopf theorem} since it was independently discovered by Andronov in 1929 and Hopf in 1943 and Poincar\'e had discussion of such result in 1892.\cite{HK}
\begin{thebibliography}{20}
\bibitem[HK]{HK} Hale, Jack H. \& Ko\,cak, H\"useyin: Dynamics and Bifurcations.  Springer-Verlag, New York, 1991.
\end{thebibliography}
%%%%%
%%%%%
\end{document}
