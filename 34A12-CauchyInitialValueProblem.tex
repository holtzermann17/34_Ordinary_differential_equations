\documentclass[12pt]{article}
\usepackage{pmmeta}
\pmcanonicalname{CauchyInitialValueProblem}
\pmcreated{2013-03-22 14:57:18}
\pmmodified{2013-03-22 14:57:18}
\pmowner{paolini}{1187}
\pmmodifier{paolini}{1187}
\pmtitle{Cauchy initial value problem}
\pmrecord{14}{36650}
\pmprivacy{1}
\pmauthor{paolini}{1187}
\pmtype{Definition}
\pmcomment{trigger rebuild}
\pmclassification{msc}{34A12}
\pmsynonym{Cauchy problem}{CauchyInitialValueProblem}
\pmsynonym{initial value problem}{CauchyInitialValueProblem}
\pmrelated{InitialValueProblem}
\pmrelated{DifferentialEquation}
\pmrelated{CauchyKowalewskiTheorem}
\pmdefines{solution to the Cauchy problem}
\pmdefines{solution to the initial value problem}

% this is the default PlanetMath preamble.  as your knowledge
% of TeX increases, you will probably want to edit this, but
% it should be fine as is for beginners.

% almost certainly you want these
\usepackage{amssymb}
\usepackage{amsmath}
\usepackage{amsfonts}

% used for TeXing text within eps files
%\usepackage{psfrag}
% need this for including graphics (\includegraphics)
%\usepackage{graphicx}
% for neatly defining theorems and propositions
%\usepackage{amsthm}
% making logically defined graphics
%%%\usepackage{xypic}

% there are many more packages, add them here as you need them

% define commands here
\newcommand{\R}{\mathbb R}
\begin{document}
Let $D$ be a subset of $\R^n\times \R$, $(x_0,t_0)$ a point of $D$, and $f\colon D\to \R$ be a function.

We say that a function $x(t)$ is a solution to the Cauchy (or initial value) problem
\begin{equation}
\begin{cases}
x'(t)=f(x(t),t)\\
x(t_0)=x_0
\end{cases}
\end{equation} 
if
\begin{enumerate}
\item $x$ is a differentiable function $x\colon I\to \R^n$ defined on a interval $I\subset \R$;
\item one has $(x(t),t)\in D$ for all $t\in I$ and $t_0\in I$;
\item one has $x(t_0)=x_0$ and $x'(t)=f(x(t),t)$ for all $t\in I$.
\end{enumerate}

We say that a solution $x\colon I\to\R^n$ is a \emph{maximal solution} if it cannot be extended to a bigger interval. More precisely given any other solution $y\colon J\to \R^n$ defined on an interval $J\supset I$ and such that $y(t)=x(t)$ for all $t\in I$, one has $I=J$ (and hence $x$ and $y$ are the same function).

We say that a solution $x\colon I\to\R^n$ is a \emph{global solution} if $D\subset=\R^n \times I$. 

We say that a solution $x\colon I\to\R^n$ is \emph{unique} if given any other solution $y\colon I\to\R^n$ one has $x(t)=y(t)$ for all $t\in I$ (i.e.\ $x$ is the unique solution defined on the interval $I$).

\subsection{Notation}
Usually the differential equation in (1) is simply written as $x'=f(x,t)$. 
Also, depending on the topics, the name chosen for the function and for the variable, can change. Other common choices are $y'=f(y,t)$ or $y'=f(y,x)$.
It is also common to write $\dot x=f(x,t)$ when the independent variable represents a time value.

\subsection{Examples}
\begin{enumerate}
\item
The function $x(t)=\log t$ defined on $I=(0,+\infty)$ is the unique maximal solution to the Cauchy problem:
\[
\begin{cases}
x'(t) = 1/t\\
x(1)=0.
\end{cases}
\]
In this case $f(x,t)=1/t$, $D=\{(x,t)\colon t\neq 0\}$, $t_0=1$, $x_0=0$.

\item
The function $x(t)=e^t$ is a global (and hence maximal), unique solution to the Cauchy problem:
\[
\begin{cases}
x'(t) = x(t)\\
x(0)=1.
\end{cases}
\]

\item
Consider the Cauchy problem
\[
\begin{cases}
x'(t) = \frac 3 2 \sqrt[3] x\\
x(0)=0.
\end{cases}
\]
The function $x(t)=0$ defined on $I=\R$ is a global solution. 
However the function $y(t)=\sqrt{t^3}$ defined on $I=[0,+\infty)$ is also a  solution and so are the functions
\[
  z(t)=\begin{cases} \sqrt{(t-c)^3}&\text{if $t\ge c$} \\ 0 &\text{if $t < c$}.\end{cases}
\]
for every $c\ge 0$.
So there are no unique solutions. Moreover $y$ is not a maximal solution.

\end{enumerate}
%%%%%
%%%%%
\end{document}
