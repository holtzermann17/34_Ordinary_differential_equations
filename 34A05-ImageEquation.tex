\documentclass[12pt]{article}
\usepackage{pmmeta}
\pmcanonicalname{ImageEquation}
\pmcreated{2014-03-20 20:16:56}
\pmmodified{2014-03-20 20:16:56}
\pmowner{pahio}{2872}
\pmmodifier{pahio}{2872}
\pmtitle{image equation }
\pmrecord{13}{88056}
\pmprivacy{1}
\pmauthor{pahio}{2872}
\pmtype{Topic}
\pmclassification{msc}{34A05}
\pmclassification{msc}{44A10}

\endmetadata

% this is the default PlanetMath preamble.  as your knowledge
% of TeX increases, you will probably want to edit this, but
% it should be fine as is for beginners.

% almost certainly you want these
\usepackage{amssymb}
\usepackage{amsmath}
\usepackage{amsfonts}

% need this for including graphics (\includegraphics)
\usepackage{graphicx}
% for neatly defining theorems and propositions
\usepackage{amsthm}

% making logically defined graphics
%\usepackage{xypic}
% used for TeXing text within eps files
%\usepackage{psfrag}

% there are many more packages, add them here as you need them

% define commands here

\begin{document}
In solving an initial value problem leading to an ordinary differential 
equation, the Laplace transform offers often a way to simplify the equation:  
both sides are Laplace transformed.\, The transformed equation, the so-called 
{\it image equation}, is in many cases simplier than the original differential 
equation, since it does not contain the derivatives of the unknown function 
$y(t)$.\, From the image equation one may solve the Laplace transform $Y(s)$ of 
$y(t)$ and then inverse transform $Y(s)$ getting $y(t)$.\\

Let's consider e.g. the ordinary $n$'th order linear 
differential equation
\begin{align}
 a_0\frac{d^ny}{dt}+a_1\frac{d^{n-1}y}{dx^{n-1}}+\ldots
 +a_{n-1}\frac{dy}{dt}+a_ny(t) \;=\; f(t)
\end{align}
subject to the initial conditions
\begin{align}
y(0)=y_0,\quad y'(0)=y_0',\quad\ldots, y^{n-1}(0)=y_0^{(n-1)}.
\end{align}
Due to the linearity of the Laplace transform the image equation 
of (1) is
\begin{align}
 a_0\mathcal{L}\{\frac{d^ny}{dx^n}\}
 +a_1\mathcal{L}\{\frac{d^{n-1}y}{dx^{n-1}}\}+\ldots
 +a_n\mathcal{L}\{y(t)\} \;=\; \mathcal{L}\{f(t)\}.
\end{align}
Denote\, $\mathcal{L}\{y(t)\} =: Y(s)$\, and\, 
$\mathcal{L}\{f(t)\} =: F(s)$.\, We put into (3) the expressions
of the Laplace transforms of the derivatives on the left hand side (see ``\PMlinkname{Laplace transforms of derivatives}{LaplaceTransformsOfDerivatives}'') 
getting
\begin{align*}
 a_0[s^nY(s)-(s^{n-1}y_0+s^{n-2}y_0'+\ldots+y_0^{(n-1)})]\\
 +a_1[s^{n-1}Y(s)-(s^{n-2}y_0+s^{n-3}y_0'+\ldots+y_0^{(n-2)})]\\
 +\ldots\qquad\ldots\qquad\ldots\qquad\ldots\\
 +a_{n-1}[sY(s)-(y_0)]\\
 = \quad F(s).
\end{align*}
This equation is simplified to
\begin{align*}
 (a_0s^n+a_1s^{n-1}+\ldots+a_{n-1}s+a_n)Y(s) \;=\\
 a_0[y_0s^{n-1}+y_0's^{n-2}+\ldots+y_0^{(n-1)}]+\\
 +a_1[y_0s^{n-2}+y_0's^{n-3}+\ldots+y_0^{(n-2)}]+\\
 +\ldots\qquad\ldots\qquad\ldots\qquad\ldots+\\
 +a_{n-2}[y_0s+y_0']+a_{n-1}[y_0]+F(s).\\
 \end{align*}
For brevity, denote in the last equation the polynomial 
multiplier of $Y(s)$ by $\varphi(s)$ and the sum preceding $F(s)$ 
by $\psi(s)$.\, Then the equation can be written as
$$\varphi(s)Y(s) \;=\; \psi(s)+F(s),$$
i.e.
\begin{align}
Y(s) \;=\; \frac{\psi(s)}{\varphi(s)}+\frac{F(s)}{\varphi(s)}.
\end{align}
The function $Y(s)$ defined by (4) is the Laplace transform of 
the solution $y(t)$ of the differential equation (1) which 
satisfies the initial conditions (2).\, If we now find a function 
$y^*(t)$ the Laplace transform of which is the function $Y(s)$ 
defined by (4), then $y^*(t)$ will do for $y(t)$ due to the 
uniqueness property of Laplace transform expressed in the entry 
``\PMlinkname{Mellin's inverse formula}{MellinsInverseFormula}''.\\
If we seek the solution of (1) satisfying the {\it zero initial 
conditions}
$$x_0 = x_0' = x_0'' = \ldots = x_0^{(n-1)} = 0,$$
then\, $\psi(s) \equiv 0$\, and
$$Y(s) \;=\; \frac{F(s)}{\varphi(s)},$$
i.e.
$$Y(s) \;=\; \frac{F(s)}{a_0s^n+a_1s^{n-1}+\ldots+a_n}.$$\\

\textbf{Example.}\, The 4'th order differential equation
\begin{align}
y''''(t)+y(t) \;=\; 0
\end{align}
should be solved with the initial conditions
$$y(0) = y'''(0) = 1, \;\;\; y'(0) = y''(0) = 0.$$
The image equation of (5) is
$$s^4Y(s)-s^3y(0)-s^2y'(0)-sy''(0)-y'''(0)+Y(s) \;=\; 0,$$
i.e.
$$(s^4\!+\!1)Y(s) \;=\; s^3\!+\!1.$$


Thus one needs to determine the inverse Laplace transform of
\begin{align}
Y(s) \;\;=\;\; \frac{1}{4}\!\cdot\!\frac{4s^3}{s^4+1}+\frac{1}{s^4+1}.
\end{align}
The zeroes of the numerator $s^4\!+\!1$ are the eighth roots of unity 
$e^{i\frac{\pi}{4}}$, $e^{i\frac{3\pi}{4}}$, $e^{i\frac{5\pi}{4}}$, 
$e^{i\frac{7\pi}{4}}$, in other words the complex numbers
$\frac{\pm 1\pm i}{\sqrt{2}}$.\, By the special case (3) of 
the Heaviside formula, the first addend of (6) 
corresponds the original function
$$\frac{1}{4}\sum_\pm e^{\frac{\pm 1\pm i}{\sqrt{2}}t} \;\;=\;\;
\frac{e^{\frac{t}{\sqrt{2}}}+e^{-\frac{t}{\sqrt{2}}}}{2}
\cdot
\frac{e^{i\frac{t}{\sqrt{2}}}+e^{-i\frac{t}{\sqrt{2}}}}{2}
\;\;=\;\;
\cosh\frac{t}{\sqrt{2}}\;\cos\frac{t}{\sqrt{2}}.$$
Utilizing also the general 
\PMlinkname{Heaviside formula}{HeavisideFormula} (1), one can 
get from (6) the result
$$y(t) \;\;:=\;\; 
\cosh{\frac{t}{\sqrt{2}}}\;\cos{\frac{t}{\sqrt{2}}}+
\frac{1}{\sqrt{2}}(\cosh{\frac{t}{\sqrt{2}}}\;\sin{\frac{t}{\sqrt{2}}}
-\sinh{\frac{t}{\sqrt{2}}}\;\cos{\frac{t}{\sqrt{2}}}).$$\\




\begin{thebibliography}{9}
\bibitem{NP}{\sc N. Piskunov:} {\em Diferentsiaal- ja 
integraalarvutus k\~{o}rgematele tehnilistele 
\~{o}ppeasutustele. Teine k\"{o}ide.} Viies tr\"{u}kk.\, 
Kirjastus Valgus, Tallinn  (1966).
\end{thebibliography}\\




\end{document}
