\documentclass[12pt]{article}
\usepackage{pmmeta}
\pmcanonicalname{TheoryForSeparationOfVariables}
\pmcreated{2013-03-22 18:37:43}
\pmmodified{2013-03-22 18:37:43}
\pmowner{pahio}{2872}
\pmmodifier{pahio}{2872}
\pmtitle{theory for separation of variables}
\pmrecord{15}{41366}
\pmprivacy{1}
\pmauthor{pahio}{2872}
\pmtype{Topic}
\pmcomment{trigger rebuild}
\pmclassification{msc}{34A09}
\pmclassification{msc}{34A05}
%\pmkeywords{separation of variables}
\pmrelated{InverseFunctionTheorem}
\pmrelated{ODETypesReductibleToTheVariablesSeparableCase}

\endmetadata

% this is the default PlanetMath preamble.  as your knowledge
% of TeX increases, you will probably want to edit this, but
% it should be fine as is for beginners.

% almost certainly you want these
\usepackage{amssymb}
\usepackage{amsmath}
\usepackage{amsfonts}

% used for TeXing text within eps files
%\usepackage{psfrag}
% need this for including graphics (\includegraphics)
%\usepackage{graphicx}
% for neatly defining theorems and propositions
 \usepackage{amsthm}
% making logically defined graphics
%%%\usepackage{xypic}

% there are many more packages, add them here as you need them

% define commands here

\theoremstyle{definition}
\newtheorem*{thmplain}{Theorem}

\begin{document}
The \PMlinkname{first order}{ODE} ordinary differential equation where one can separate the variables has the form where $\displaystyle\frac{dy}{dx}$ may be expressed as \PMlinkname{a product or a quotient of two functions}{ProductOfFunctions}, one of which depends only on $x$ and the other on $y$.\, Such an equation may be written e.g. as
\begin{align}
\frac{dy}{dx} \;=\; \frac{Y(y)}{X(x)} \quad \mbox{or} \quad \frac{dx}{dy} \;=\; \frac{X(x)}{Y(y)}.
\end{align}


We notice first that if $Y(y)$ has real \PMlinkname{zeroes}{ZeroOfAFunction} $y_1,\,y_2,\,\ldots$, then the equation (1) has the constant solutions \,$y := y_1,\; y := y_2,\; \ldots$\; and thus the lines \,$y = y_1,\; y = y_2,\; \ldots$\; are integral curves.\, Similarly, if $X(x)$ has real zeroes \,$x_1,\,x_2,\,\ldots$, one has to include the lines\, $y = y_1,\; y = y_2,\; \ldots$\; to the integral curves.\, All those lines \PMlinkescapetext{divide} the $xy$-plane into the rectangular regions.\, One can obtain other integral curves only inside such regions where the derivative $\displaystyle\frac{dy}{dx}$ attains real values.

Let $R$ be such a region, defined by
$$a < x < b, \quad c < y < d,$$
and let us assume that the $X(x)$ and $Y(y)$ are real, continuous and distinct from zero in $R$.\, We will show that any integral curve of the differential equation (1) is accessible by two quadratures.\\

Let $\gamma$ be an integral curve passing through the point \,$(x_0,\,y_0)$\, of the region $R$.\, By the above assumptions, the derivative $\displaystyle\frac{dy}{dx}$ maintains its sign on the curve $\gamma$ so long $\gamma$ is inside $R$, which is true on a neighbourhood $N$ of \,$x_0$, contained in\, $[a,\,b]$.\, This implies that as $x$ runs the interval \,$N$,\, it defines the ordinate $y$ of $\gamma$ uniquely as a monotonic function \,$y \mapsto y(x)$\, which satisfies the equation (1):
$$y'(x) \;=\; \frac{Y(y(x))}{X(x)}$$
The last equation may be written 
\begin{align}
\frac{y'(x)}{Y(y(x))} \,=\, \frac{1}{X(x)}.
\end{align}
Since $X$ and $Y$ don't vanish in $R$, the denominators $Y(y(x))$ and $X(x)$ are distinct from 0 on the interval $N$.\, Therefore one can integrate both sides of (2) from $x_0$ to an arbitrary value $x$ on $N$, getting
\begin{align}
\int_{x_0}^x\frac{y'(x)\,dx}{Y(y(x))} \,=\, \int_{x_0}^x\frac{dx}{X(x)}.
\end{align}
Because \,$y = y(x)$\, is continuous and monotonic on the interval $N$, it can be taken as \PMlinkname{new variable of integration}{SubstitutionForIntegration} in the left hand side of (3): \,substitute\, $y(x) := y$,\; $y'(x)\,dx := dy$\, and change the \PMlinkescapetext{limits} to\, $y(x_0) = y_0$\, and\, $y(x) = y$.\\

\begin{itemize}

\item Accordingly, the equality
\begin{align}
\int_{y_0}^y\frac{dy}{Y(y)} \;=\; \int_{x_0}^x\frac{dx}{X(x)}
\end{align}
is valid, meaning that if an integral curve of (1) passes through the point \,$(x_0,\,y_0)$, the integral curve is represented by the equation (4) as long as the curve is inside the region $R$.\\

\item Additionally, it is possible to justificate that if\, $(x_0,\,y_0)$\, is an interior point of a region $R$ where $X(x)$ and $Y(y)$ are real, continuous and $\neq 0$, then one and only one integral curve of (1) passes through this point, the curve is \PMlinkname{regular}{RegularCurve}, and both $x$ and $y$ are monotonic on it.\, N.B., the Lipschitz condition for the right hand side of (1) is not necessary for the justification.

\item When the point\, $(x_0,\,y_0)$\, changes in the region $R$, (4) gives a family of integral curves which cover the region once.\, The equations of these curves may be unified to the form 
\begin{align}
\int\frac{dy}{Y(y)} \;=\; \int\frac{dx}{X(x)},
\end{align}
which thus \PMlinkescapetext{represents} the general solution of the differential equation (1) in $R$.\, Hence one can speak of the {\em separation of variables},
\begin{align}
\frac{dy}{Y(y)} \;=\; \frac{dx}{X(x)},
\end{align}
and integration of both sides.

\end{itemize}

\begin{thebibliography}{9}
\bibitem{3L}{\sc E. Lindel\"of:} {\em Differentiali- ja integralilasku III 1}.\, Mercatorin Kirjapaino Osakeyhti\"o, Helsinki (1935).
\end{thebibliography}

%%%%%
%%%%%
\end{document}
