\documentclass[12pt]{article}
\usepackage{pmmeta}
\pmcanonicalname{ChebyshevEquation}
\pmcreated{2013-03-22 13:10:17}
\pmmodified{2013-03-22 13:10:17}
\pmowner{mclase}{549}
\pmmodifier{mclase}{549}
\pmtitle{Chebyshev equation}
\pmrecord{6}{33616}
\pmprivacy{1}
\pmauthor{mclase}{549}
\pmtype{Definition}
\pmcomment{trigger rebuild}
\pmclassification{msc}{34A30}
\pmsynonym{Chebyshev differential equation}{ChebyshevEquation}
\pmrelated{HermiteEquation}

% this is the default PlanetMath preamble.  as your knowledge
% of TeX increases, you will probably want to edit this, but
% it should be fine as is for beginners.

% almost certainly you want these
\usepackage{amssymb}
\usepackage{amsmath}
\usepackage{amsfonts}

% used for TeXing text within eps files
%\usepackage{psfrag}
% need this for including graphics (\includegraphics)
%\usepackage{graphicx}
% for neatly defining theorems and propositions
%\usepackage{amsthm}
% making logically defined graphics
%%%\usepackage{xypic}

% there are many more packages, add them here as you need them

% define commands here
\begin{document}
Chebyshev's equation is the second order linear differential equation
$$(1-x^2)\frac{d^2y}{dx^2} - x\frac{dy}{dx} + p^2y = 0$$
where $p$ is a real constant.

There are two independent solutions which are given as series by:

$$
y_1(x) = 1 - \tfrac{p^2}{2!}x^2 + \tfrac{(p-2)p^2(p+2)}{4!}x^4
      - \tfrac{(p-4)(p-2)p^2(p+2)(p+4)}{6!}x^6 + \dotsb
$$
and
$$
y_2(x) = x - \tfrac{(p-1)(p+1)}{3!}x^3 + \tfrac{(p-3)(p-1)(p+1)(p+3)}{5!}x^5 - \dotsb
$$

In each case, the coefficients are given by the recursion
$$
a_{n+2} = \frac{(n-p)(n+p)}{(n+1)(n+2)} a_n
$$
with $y_1$ arising from the choice $a_0 = 1$, $a_1 = 0$,
and $y_2$ arising from the choice $a_0 = 0$, $a_1 = 1$.

The series converge for $|x| < 1$; this is easy to see from the ratio test and the recursion formula above.

When $p$ is a non-negative integer, one of these series will terminate,
giving a polynomial solution.
If $p \ge 0$ is even, then the series for $y_1$ terminates at $x^p$.
If $p$ is odd, then the series for $y_2$ terminates at $x^p$.

These polynomials are, up to multiplication by a constant, the Chebyshev polynomials.  These are the only polynomial solutions of the Chebyshev equation.

(In fact, polynomial solutions are also obtained when $p$ is a negative integer,
but these are not new solutions, since the Chebyshev equation is invariant under the substitution of $p$ by $-p$.)
%%%%%
%%%%%
\end{document}
