\documentclass[12pt]{article}
\usepackage{pmmeta}
\pmcanonicalname{DifferentialEquationOfCircles}
\pmcreated{2013-03-22 18:59:26}
\pmmodified{2013-03-22 18:59:26}
\pmowner{pahio}{2872}
\pmmodifier{pahio}{2872}
\pmtitle{differential equation of circles}
\pmrecord{5}{41857}
\pmprivacy{1}
\pmauthor{pahio}{2872}
\pmtype{Example}
\pmcomment{trigger rebuild}
\pmclassification{msc}{34A34}
\pmclassification{msc}{51-00}
%\pmkeywords{parametre}

\endmetadata

% this is the default PlanetMath preamble.  as your knowledge
% of TeX increases, you will probably want to edit this, but
% it should be fine as is for beginners.

% almost certainly you want these
\usepackage{amssymb}
\usepackage{amsmath}
\usepackage{amsfonts}

% used for TeXing text within eps files
%\usepackage{psfrag}
% need this for including graphics (\includegraphics)
%\usepackage{graphicx}
% for neatly defining theorems and propositions
 \usepackage{amsthm}
% making logically defined graphics
%%%\usepackage{xypic}

% there are many more packages, add them here as you need them

% define commands here

\theoremstyle{definition}
\newtheorem*{thmplain}{Theorem}

\begin{document}
All circles of the plane form a three-parametric family
$$(x-a)^2+(y-b)^2 \;=\; r^2.$$
The parametres $a,\,b,\,r$ may be eliminated by using successive differentiations, when one gets
$$x-a+(y-b)y' \;=\; 0,$$ 
$$1+y'^{\,2}+(y-b)y'' = 0,$$
$$3y'y''+(y-b)y''' \;=\; 0.$$
The two last equations allow to eliminate also $b$, yielding
the differential equation of all circles of the plane:
$$(1+y'^{\,2})y'''-3y'y''^{\,2} \;=\; 0$$
It is of \PMlinkescapetext{order} three, corresponding the number of parametres.

%%%%%
%%%%%
\end{document}
