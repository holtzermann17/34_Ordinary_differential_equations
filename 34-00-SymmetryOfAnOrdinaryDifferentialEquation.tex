\documentclass[12pt]{article}
\usepackage{pmmeta}
\pmcanonicalname{SymmetryOfAnOrdinaryDifferentialEquation}
\pmcreated{2013-03-22 13:42:24}
\pmmodified{2013-03-22 13:42:24}
\pmowner{Daume}{40}
\pmmodifier{Daume}{40}
\pmtitle{symmetry of an ordinary differential equation}
\pmrecord{10}{34384}
\pmprivacy{1}
\pmauthor{Daume}{40}
\pmtype{Definition}
\pmcomment{trigger rebuild}
\pmclassification{msc}{34-00}
\pmsynonym{symmetry of an differential equation}{SymmetryOfAnOrdinaryDifferentialEquation}

\endmetadata

% this is the default PlanetMath preamble.  as your knowledge
% of TeX increases, you will probably want to edit this, but
% it should be fine as is for beginners.

% almost certainly you want these
\usepackage{amssymb}
\usepackage{amsmath}
\usepackage{amsfonts}

% used for TeXing text within eps files
%\usepackage{psfrag}
% need this for including graphics (\includegraphics)
%\usepackage{graphicx}
% for neatly defining theorems and propositions
%\usepackage{amsthm}
% making logically defined graphics
%%%\usepackage{xypic} 

% there are many more packages, add them here as you need them

% define commands here
\begin{document}
Let $ f:\mathbb{R}^n \to \mathbb{R}^n$ be a smooth function and let
$$\dot{x} = f(x)$$
be a system of ordinary differential equations, in addition let $\gamma$ be an invertible matrix.  Then $\gamma$ is a \emph{\PMlinkescapetext{symmetry}\ of the ordinary differential equation} if
$$f(\gamma x) = \gamma f(x).$$\\
\textbf{Example:}
\begin{itemize}
\item Natural symmetry of the Lorenz equation is a \PMlinkescapetext{simple} example of a symmetry of a differential equation.
\end{itemize}
\begin{thebibliography}{1}
\bibitem[GSS]{1} Golubitsky, Martin. Stewart, Ian. Schaeffer, G. David: Singularities and Groups in Bifurcation Theory \textit{(Volume II)}. Springer-Verlag, New York, 1988.
\end{thebibliography}
%%%%%
%%%%%
\end{document}
