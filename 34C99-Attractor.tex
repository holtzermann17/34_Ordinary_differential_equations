\documentclass[12pt]{article}
\usepackage{pmmeta}
\pmcanonicalname{Attractor}
\pmcreated{2013-03-22 15:17:34}
\pmmodified{2013-03-22 15:17:34}
\pmowner{Daume}{40}
\pmmodifier{Daume}{40}
\pmtitle{attractor}
\pmrecord{4}{37089}
\pmprivacy{1}
\pmauthor{Daume}{40}
\pmtype{Definition}
\pmcomment{trigger rebuild}
\pmclassification{msc}{34C99}
\pmdefines{attracting set}
\pmdefines{repelling set}
\pmdefines{repellor}

% this is the default PlanetMath preamble.  as your knowledge
% of TeX increases, you will probably want to edit this, but
% it should be fine as is for beginners.

% almost certainly you want these
\usepackage{amssymb}
\usepackage{amsmath}
\usepackage{amsfonts}
\usepackage{amsthm}

% used for TeXing text within eps files
%\usepackage{psfrag}
% need this for including graphics (\includegraphics)
%\usepackage{graphicx}
% making logically defined graphics
%%%\usepackage{xypic} 

% there are many more packages, add them here as you need them

% define commands here

% The below lines should work as the command
% \renewcommand{\bibname}{References}
% without creating havoc when rendering an entry in
% the page-image mode.
\makeatletter
\@ifundefined{bibname}{}{\renewcommand{\bibname}{References}}
\makeatother

\newtheorem{thm}{Theorem}
\newtheorem{defn}{Definition}
\newtheorem{prop}{Proposition}
\newtheorem{lemma}{Lemma}
\newtheorem{cor}{Corollary}
\begin{document}
Let
$$\dot{x}=f(x)$$
be a system of autonomous ordinary differential equation in $\mathbb{R}^n$ defined by a vector field $f\colon \mathbb{R}^n\to \mathbb{R}^n$.
A set $A$ is said to be an \emph{attracting set}\cite{GH,P} if
\begin{enumerate}
\item $A$ is closed and invariant,
\item there exists an open neighborhood $U$ of $A$ such that all solution 
with initial solution in $U$ will eventually enter $A$ ($x(t)\to A$) as $t\to \infty$. 
\end{enumerate}
Additionally, if $A$ contains a dense orbit then $A$ is said to be an \emph{attractor}\cite{GH,P}.\\
Conversely, a set $R$ is said to be a \emph{repelling set}\cite{GH} if $R$ satisfy the condition 1. and  2. where $t\to \infty$ is replaced by $t\to -\infty$.  Similarly, if $R$ contains a dense orbit then $R$ is said to be a \emph{repellor}\cite{GH}.

\begin{thebibliography}{1}
\bibitem[GH]{GH}
{\scshape Guckenheimer, John \& Holmes, Philip},
\emph{Nonlinear Oscillations, Dynamical Systems, 
and Bifurcations of Vector Fields}, 
Springer, New York, 1983.
\bibitem[P]{P}
{\scshape Perko, Lawrence},
\emph{Differential Equations and Dynamical Systems}, 
Springer, New York, 2001.
\end{thebibliography}
%%%%%
%%%%%
\end{document}
