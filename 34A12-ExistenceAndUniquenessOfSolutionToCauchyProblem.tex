\documentclass[12pt]{article}
\usepackage{pmmeta}
\pmcanonicalname{ExistenceAndUniquenessOfSolutionToCauchyProblem}
\pmcreated{2013-03-22 16:54:45}
\pmmodified{2013-03-22 16:54:45}
\pmowner{ehremo}{15714}
\pmmodifier{ehremo}{15714}
\pmtitle{existence and uniqueness of solution to Cauchy problem}
\pmrecord{20}{39173}
\pmprivacy{1}
\pmauthor{ehremo}{15714}
\pmtype{Theorem}
\pmcomment{trigger rebuild}
\pmclassification{msc}{34A12}
%\pmkeywords{cauchy initial value problem existence uniqueness solution}

% this is the default PlanetMath preamble.  as your knowledge
% of TeX increases, you will probably want to edit this, but
% it should be fine as is for beginners.

% almost certainly you want these
\usepackage{amssymb}
\usepackage{amsmath}
\usepackage{amsfonts}

% used for TeXing text within eps files
%\usepackage{psfrag}
% need this for including graphics (\includegraphics)
%\usepackage{graphicx}
% for neatly defining theorems and propositions
%\usepackage{amsthm}
% making logically defined graphics
%%%\usepackage{xypic}

% there are many more packages, add them here as you need them

% define commands here

\def\reals{\mathbb{R}}
\def\v#1{\mathbf{#1}}
\def\vdot#1{\mathbf{\dot{#1}}}
\def\d{\mathrm{d}}
\def\norm#1{\|#1\|}
\def\eps{\epsilon}
\begin{document}
Let
$$\begin{cases}\vdot x = F(\v x, t) \\ \v x(t_0) = \v x_0\end{cases}$$
be a Cauchy problem, where $F : U \to \reals$ is
\begin{itemize}
\item a continuous function of $n+1$ variables defined in a neighborhood $U \subseteq \reals^{n+1}$ of $(\v x_0, t_0)$
\item Lipschitz continuous with respect to the first $n$ variables (i.e. with respect to $\v x$).
\end{itemize}

Then there exists a unique solution $\v x : I \to \reals^n$ of the Cauchy problem, defined in a neighborhood $I \subseteq \reals$ of $t_0$.

{\bf Proof}

Solving the Cauchy problem is equivalent to solving the following integral equation
$$x(t) = x(t_0) + \int_{t_0}^{t} F(\v x(\tau), \tau) \d\tau$$

Let $X$ be the set of continuous functions $\v f : [t_0 - \delta, t_0 + \delta] \to B(\v x_0, \eps)$. We'll assume $\eps$ to be chosen such that the $B(\v x_0, \eps) \subseteq U$ \footnote{$B(\v x_0, \eps)$ denotes the closed ball $\{ \v x : \norm{\v x_0 - \v x} \leq \eps \}$}. In this ball, therefore, $F$ is Lipschitz continuous with respect to the first $n$ variable, in other words, there exists a real number $L$ such that
$${F(\v x, t) - F(\v y, t)} \leq L\norm{\v x - \v y}$$
for all points $\v x, \v y$ sufficiently near to $\v x_0$.

Now let's define the mapping $T : X \to X$ as follows
$$T\v x : t \mapsto \v x_0 + \int_{t_0}^t F(\v x(\tau), \tau) \d \tau$$
We make the following observations about $T$.
\begin{enumerate}
\item Since $F$ is continuous, $\norm{F}$ attains a maximum value $M$ on the compact set $B(\v x_0, \eps) \times [t_0 \pm \delta]$. But by hypothesis, $\norm{\v x(t) - \v x_0} \leq \eps$, hence
$$\norm{\v x(t) -  \v x_0} \leq \int_{t_0}^t \norm{F(\v x(\tau), \tau)} \d \tau \leq M(t - t_0) \leq M\delta$$
for all $t \in [t_0 \pm \delta]$.
\item The Lipschitz continuity of $F$ yields
$$\norm{T\v x(t) - T\v y(t)} \leq \int_{t_0}^t \norm{F(\v x(\tau), \tau) - F(\v y(\tau), \tau)} \d \tau \leq \int_{t_0}^t L\norm{\v x(\tau) - \v y(\tau)} \d \tau \leq L \delta d_\infty(\v x, \v y)$$
\end{enumerate}
If we choose $\delta < \min\{1/L, \eps/M\}$ these conditions ensure that 
\begin{itemize}
\item $T(X) \subseteq X$, i.e. $T$ doesn't send us outside of $X$.
\item $T$ is a contraction mapping with respect to the uniform convergence metric $d_\infty$ on $X$, i.e. there exists $\lambda \in \reals$ such that for all $\v x, \v y \in X$,
$$d_\infty(T\v x, T\v y) \leq \lambda d_\infty(\v x, \v x)$$
\end{itemize}
In particular, the second point allows us to apply Banach's theorem and define
$$\v x^\star = \lim_{k \to \infty} T^k\v x_0$$
to find the unique fixed point of $T$ in $X$, i.e. the unique function which solves
$$T\v x = \v x \text{ in other words } \v x(t) = \v x_0 + \int_{t_0}^t F(\v x(\tau), \tau) \d \tau$$
and which therefore locally solves the Cauchy problem.
%%%%%
%%%%%
\end{document}
