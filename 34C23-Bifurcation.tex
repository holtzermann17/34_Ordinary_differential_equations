\documentclass[12pt]{article}
\usepackage{pmmeta}
\pmcanonicalname{Bifurcation}
\pmcreated{2013-03-22 12:34:21}
\pmmodified{2013-03-22 12:34:21}
\pmowner{CWoo}{3771}
\pmmodifier{CWoo}{3771}
\pmtitle{bifurcation}
\pmrecord{11}{32821}
\pmprivacy{1}
\pmauthor{CWoo}{3771}
\pmtype{Definition}
\pmcomment{trigger rebuild}
\pmclassification{msc}{34C23}
\pmclassification{msc}{35B32}
\pmclassification{msc}{37H20}
%\pmkeywords{bifurcation}
%\pmkeywords{dynamical systems}
%\pmkeywords{catastrophe theory}
\pmrelated{DynamicalSystem}
\pmrelated{SystemDefinitions}

\usepackage{amssymb}
\usepackage{amsmath}
\usepackage{amsfonts}

%\usepackage{psfrag}
%\usepackage{graphicx}
%%%\usepackage{xypic}
\begin{document}
\emph{Bifurcation} refers to the splitting of dynamical systems. The parameter space of a dynamical system is regular if all points in the sufficiently small open neighborhood correspond to the dynamical systems that are equivalent to this one; a parameter point that is not regular is a bifurcation point.

For example, the branching of the Feigenbaum tree is an instance of bifurcation.

A cascade of bifurcations is a precursor to chaotic dynamics.
The topologist Ren\'e Thom in his book on catastrophe theory in biology discusses the cusp bifurcation
as a basic example of (dynamic) `catastrophe' in morphogenesis and biological development. 

\begin{thebibliography}{3}

\bibitem{B1} ``Bifurcations'', \PMlinkexternal{http://mcasco.com/bifurcat.html}{http://mcasco.com/bifurcat.html}

\bibitem{B2} ``Bifurcation'', \PMlinkexternal{http://spanky.triumf.ca/www/fractint/bif_type.html}{http://spanky.triumf.ca/www/fractint/bif_type.html}

\bibitem{B3} ``Quadratic Iteration, bifurcation, and chaos'', \PMlinkexternal{http://mathforum.org/advanced/robertd/bifurcation.html}{http://mathforum.org/advanced/robertd/bifurcation.html}

\end{thebibliography}
%%%%%
%%%%%
\end{document}
