\documentclass[12pt]{article}
\usepackage{pmmeta}
\pmcanonicalname{NaturalSymmetryOfTheLorenzEquation}
\pmcreated{2013-03-22 13:44:12}
\pmmodified{2013-03-22 13:44:12}
\pmowner{Daume}{40}
\pmmodifier{Daume}{40}
\pmtitle{natural symmetry of the Lorenz equation}
\pmrecord{5}{34428}
\pmprivacy{1}
\pmauthor{Daume}{40}
\pmtype{Result}
\pmcomment{trigger rebuild}
\pmclassification{msc}{34-00}
\pmclassification{msc}{65P20}
\pmclassification{msc}{65P30}
\pmclassification{msc}{65P40}
\pmclassification{msc}{65P99}

% this is the default PlanetMath preamble.  as your knowledge
% of TeX increases, you will probably want to edit this, but
% it should be fine as is for beginners.

% almost certainly you want these
\usepackage{amssymb}
\usepackage{amsmath}
\usepackage{amsfonts}

% used for TeXing text within eps files
%\usepackage{psfrag}
% need this for including graphics (\includegraphics)
%\usepackage{graphicx}
% for neatly defining theorems and propositions
%\usepackage{amsthm}
% making logically defined graphics
%%%\usepackage{xypic} 

% there are many more packages, add them here as you need them

% define commands here
\begin{document}
The Lorenz equation has a natural symmetry defined by 
\begin{equation}
(x,y,z) \mapsto (-x,-y,z). \label{eq:sym}
\end{equation}
To verify that (\ref{eq:sym}) is a symmetry of an ordinary differential equation (Lorenz equation) there must exist a $3\times3$ matrix which commutes with the differential equation.  This can be easily verified by observing that the symmetry is associated with the matrix $R$ defined as
\begin{equation}
R = \begin{bmatrix}
-1 & 0 & 0  \\
0 & -1 & 0 \\
0 & 0 & 1 
\end{bmatrix}.
\end{equation}
Let
\begin{equation}
\dot{\textbf{x}} = f(\textbf{x}) = \begin{bmatrix}
\sigma(y-x)  \\
x(\tau - z) -y \\
xy - \beta z
\end{bmatrix}
\end{equation}
where $f(\textbf{x})$ is the Lorenz equation and $\textbf{x}^T = (x,y,z)$.  We proceed by showing that $Rf(\textbf{x}) = f(R\textbf{x})$. Looking at the left hand side
\begin{eqnarray*}
Rf(\textbf{x}) & = & \begin{bmatrix}
-1 & 0 & 0  \\
0 & -1 & 0 \\
0 & 0 & 1 
\end{bmatrix} 
\begin{bmatrix}
\sigma(y-x)  \\
x(\tau - z) -y \\
xy - \beta z
\end{bmatrix}\\
& = & \begin{bmatrix}
\sigma(x-y)  \\
x(z - \tau ) + y \\
xy - \beta z
\end{bmatrix}
\end{eqnarray*}
and now looking at the right hand side
\begin{eqnarray*}
f(R\textbf{x}) & = & f(\begin{bmatrix}
-1 & 0 & 0 \\
0 & -1 & 0 \\
0 & 0 & 1 
\end{bmatrix}\begin{bmatrix}
x \\
y \\
z
\end{bmatrix})\\
& = & f(\begin{bmatrix}
-x \\
-y \\
z
\end{bmatrix})\\
& = & \begin{bmatrix}
\sigma(x-y)  \\
x(z - \tau ) + y \\
xy - \beta z
\end{bmatrix}.
\end{eqnarray*}
Since the left hand side is equal to the right hand side then (\ref{eq:sym}) is a symmetry of the Lorenz equation.
%%%%%
%%%%%
\end{document}
