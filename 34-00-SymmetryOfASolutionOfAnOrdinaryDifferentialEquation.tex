\documentclass[12pt]{article}
\usepackage{pmmeta}
\pmcanonicalname{SymmetryOfASolutionOfAnOrdinaryDifferentialEquation}
\pmcreated{2013-03-22 13:42:26}
\pmmodified{2013-03-22 13:42:26}
\pmowner{Daume}{40}
\pmmodifier{Daume}{40}
\pmtitle{symmetry of a solution of an ordinary differential equation}
\pmrecord{11}{34385}
\pmprivacy{1}
\pmauthor{Daume}{40}
\pmtype{Definition}
\pmcomment{trigger rebuild}
\pmclassification{msc}{34-00}
\pmsynonym{symmetry of a periodic solution solution of an ordinary differential equation}{SymmetryOfASolutionOfAnOrdinaryDifferentialEquation}

% this is the default PlanetMath preamble.  as your knowledge
% of TeX increases, you will probably want to edit this, but
% it should be fine as is for beginners.

% almost certainly you want these
\usepackage{amssymb}
\usepackage{amsmath}
\usepackage{amsfonts}

% used for TeXing text within eps files
%\usepackage{psfrag}
% need this for including graphics (\includegraphics)
%\usepackage{graphicx}
% for neatly defining theorems and propositions
%\usepackage{amsthm}
% making logically defined graphics
%%%\usepackage{xypic} 

% there are many more packages, add them here as you need them

% define commands here

% The below lines should work as the command
% \renewcommand{\bibname}{References}
% without creating havoc when rendering an entry in
% the page-image mode.
\makeatletter
\@ifundefined{bibname}{}{\renewcommand{\bibname}{References}}
\makeatother
\begin{document}
\PMlinkescapeword{lemma}

Let $\gamma$ be a \PMlinkname{symmetry of the ordinary differential equation}{SymmetryOfAnOrdinaryDifferentialEquation} and $x_0$ be a steady state solution of $\dot{x} = f(x)$.  If $$\gamma x_0 = x_0$$ then $\gamma$ is called a \emph{symmetry of the solution of $x_0$}.\\

Let $\gamma$ be a symmetry of the ordinary differential equation and $x_0(t)$ be a periodic solution of $\dot{x} = f(x)$.  If $$\gamma x_0(t-t_0) = x_0(t)$$ for a certain $t_0$ then $(\gamma, t_0)$ is called a \emph{symmetry of the periodic solution of $x_0(t)$}.\\\\
\textbf{lemma:} If $\gamma$ is a symmetry of the ordinary differential equation and let $x_0(t)$ be a solution\textit{(either steady state or periodic)} of $\dot{x} = f(x)$.  Then $\gamma x_0(t)$ is a solution of $\dot{x} = f(x)$.\\
\textit{proof:}  If $x_0(t)$ is a solution of $\frac{d x}{dt} = f(x)$ implies $\frac{d x_0(t)}{dt} = f(x_0(t))$.  Let's now verify that $\gamma x_0(t)$ is a solution, with a substitution into $\frac{d x}{dt} = f(x)$.  The left hand side of the equation becomes $\frac{d\gamma x_0(t)}{dt} = \gamma \frac{dx_0(t)}{dt}$ and the right hand side of the equation becomes $f(\gamma x_0(t))= \gamma f(x_0(t))$ since $\gamma$ is a symmetry of the differential equation.  Therefore we have that the left hand side equals the right hand side since $\frac{dx_0(t)}{dt} = f(x_0(t))$.
qed
\begin{thebibliography}{1}
\bibitem[GSS]{1} Golubitsky, Martin. Stewart, Ian. Schaeffer, G. David: Singularities and Groups in Bifurcation Theory \textit{(Volume II)}. Springer-Verlag, New York, 1988.
\end{thebibliography}
%%%%%
%%%%%
\end{document}
