\documentclass[12pt]{article}
\usepackage{pmmeta}
\pmcanonicalname{ODETypesReductibleToTheVariablesSeparableCase}
\pmcreated{2013-03-22 18:06:36}
\pmmodified{2013-03-22 18:06:36}
\pmowner{pahio}{2872}
\pmmodifier{pahio}{2872}
\pmtitle{ODE types reductible to the variables separable case}
\pmrecord{13}{40655}
\pmprivacy{1}
\pmauthor{pahio}{2872}
\pmtype{Topic}
\pmcomment{trigger rebuild}
\pmclassification{msc}{34A09}
\pmclassification{msc}{34A05}
\pmrelated{SeparationOfVariables}
\pmrelated{ODETypesSolvableByTwoQuadratures}
\pmrelated{TheoryForSeparationOfVariables}
\pmdefines{homogeneous differential equation}
\pmdefines{similarity equation}

\endmetadata

% this is the default PlanetMath preamble.  as your knowledge
% of TeX increases, you will probably want to edit this, but
% it should be fine as is for beginners.

% almost certainly you want these
\usepackage{amssymb}
\usepackage{amsmath}
\usepackage{amsfonts}

% used for TeXing text within eps files
%\usepackage{psfrag}
% need this for including graphics (\includegraphics)
%\usepackage{graphicx}
% for neatly defining theorems and propositions
 \usepackage{amsthm}
% making logically defined graphics
%%%\usepackage{xypic}

% there are many more packages, add them here as you need them

% define commands here

\theoremstyle{definition}
\newtheorem*{thmplain}{Theorem}

\begin{document}
There are certain \PMlinkescapetext{types} of non-linear ordinary differential equations of \PMlinkname{first order}{ODE} which may by a suitable substitution be \PMlinkescapetext{reduced} to a form where one can \PMlinkname{separate}{SeparationOfVariables} the variables.\\

\textbf{I.\; So-called homogeneous differential equation}

This means the equation of the form
$$X(x,\,y)dx+Y(x,\,y)dy = 0,$$
where $X$ and $Y$ are two homogeneous functions of the same \PMlinkname{degree}{HomogeneousFunction}.\, Therefore, if the equation is written as
$$\frac{dy}{dx} = -\frac{X(x,\,y)}{Y(x,\,y)},$$
its right hand side is a homogeneous function of degree 0, i.e. it depends only on the ratio $y\!:\!x$, and has thus the form
\begin{align}
\frac{dy}{dx} = f\left(\frac{y}{x}\right).
\end{align}

Accordingly, if this ratio is constant, then also $\frac{dy}{dx}$ is constant; thus all lines \, $\frac{y}{x} =$ constant\, are isoclines of the family of the integral curves which intersect any such line isogonally.

We can infer as well, that if one integral curve is represented by\, $x = x(t)$,\; $y = y(t)$,\, then also\, $x = Cx(t)$,\; $y = Cy(t)$\, \PMlinkescapetext{represents} an integral curve for any constant $C$.\, Hence the integral curves are homothetic with respect to the origin; therefore some people call the equation (1) a {\em similarity equation}.

For generally solving the equation (1), make the substitution
$$\frac{y}{x} := t; \quad y = tx; \quad \frac{dy}{dx} = t+x\frac{dt}{dx}.$$
The equation takes the form
\begin{align}
t+x\frac{dt}{dx} = f(t)
\end{align}
which shows that any \PMlinkname{root}{Equation} $t_\nu$ of the equality \,$f(t) = t$\, gives a singular solution \, $y = t_\nu x$.
The variables in (2) may be \PMlinkescapetext{separated}:
$$\frac{dx}{x} = \frac{dt}{f(t)\!-\!t}$$
Thus one obtains\, $\ln{|x|} = \int\!\frac{dt}{f(t)\!-\!t}+ \ln{C}$, whence the general solution of the homogeneous differential equation (1) is in a parametric form
$$x = Ce^{\int\!\frac{dt}{f(t)\!-\!t}}, \quad y = Cte^{\int\!\frac{dt}{f(t)\!-\!t}}.$$



\textbf{II.\; Equation of the form\, {\em y}$\,'${\em = f(ax+by+c)}}

It's a question of the equation
\begin{align}
\frac{dy}{dx} = f(ax+by+c),
\end{align}
where $a$, $b$ and $c$ are given constants.\, If\, $ax+by$ is constant, then $\frac{dy}{dx}$ is constant, and we see that the lines\, $ax+by =$ constant\, are isoclines of the intgral curves of (3).

Let
\begin{align}
ax+by+c := u
\end{align}
be a new variable.\, It changes the equation (3) to
\begin{align}
\frac{du}{dx} = a+bf(u).
\end{align}
Here, one can see that the real zeros $u$ of the right hand side yield lines (4) which are integral curves of (3), and thus we have singular solutions.\, Moreover, one can separate the variables in (5) and integrate, obtaining $x$ as a function of $u$.\, Using still (4) gives also $y$.\, The general solution is
$$x = \int\frac{du}{a+bf(u)}+C, \quad y = \frac{1}{b}\left(u-c-a\int\frac{du}{a+bf(u)}-aC\right).\\$$


\textbf{Example.}\, In the nonlinear equation
$$\frac{dy}{dx} = (x-y)^2,$$
which is of the type II, one cannot separate the variables $x$ and $y$.\, The substitution\, $x-y := u$\, converts it to
$$\frac{du}{dx} = 1-u^2,$$
where one can separate the variables.\, Since the right hand side has the zeros\, $u = \pm1$,\, the given equation has the singular solutions $y$ given by\, $x-y = \pm1$.\, Separating the variables $x$ and $u$, one obtains
$$dx = \frac{du}{1-u^2},$$
whence
$$x = \int\frac{du}{(1+u)(1-u)} = \frac{1}{2}\int\left(\frac{1}{1+u}+\frac{1}{1-u}\right)du 
= \frac{1}{2}\ln\left|\frac{1+u}{1-u}\right|+C.$$
Accordingly, the given differential equation has the parametric solution
$$x = \ln\sqrt{\left|\frac{1\!+\!u}{1\!-\!u}\right|}+C, \quad y = \ln\sqrt{\left|\frac{1\!+\!u}{1\!-\!u}\right|}-u\!+\!C.$$

\begin{thebibliography}{9}
\bibitem{3L}{\sc E. Lindel\"of:} {\em Differentiali- ja integralilasku III 1}.\, Mercatorin Kirjapaino Osakeyhti\"o, Helsinki (1935).
\end{thebibliography}

%%%%%
%%%%%
\end{document}
