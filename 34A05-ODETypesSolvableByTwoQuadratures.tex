\documentclass[12pt]{article}
\usepackage{pmmeta}
\pmcanonicalname{ODETypesSolvableByTwoQuadratures}
\pmcreated{2015-03-20 17:04:58}
\pmmodified{2015-03-20 17:04:58}
\pmowner{pahio}{2872}
\pmmodifier{pahio}{2872}
\pmtitle{ODE types solvable by two quadratures}
\pmrecord{13}{41329}
\pmprivacy{1}
\pmauthor{pahio}{2872}
\pmtype{Topic}
\pmcomment{trigger rebuild}
\pmclassification{msc}{34A05}
\pmsynonym{second order ODE types solvable by quadratures}{ODETypesSolvableByTwoQuadratures}
\pmrelated{ODETypesReductibleToTheVariablesSeparableCase}

\endmetadata

% this is the default PlanetMath preamble.  as your knowledge
% of TeX increases, you will probably want to edit this, but
% it should be fine as is for beginners.

% almost certainly you want these
\usepackage{amssymb}
\usepackage{amsmath}
\usepackage{amsfonts}

% used for TeXing text within eps files
%\usepackage{psfrag}
% need this for including graphics (\includegraphics)
%\usepackage{graphicx}
% for neatly defining theorems and propositions
 \usepackage{amsthm}
% making logically defined graphics
%%%\usepackage{xypic}

% there are many more packages, add them here as you need them

% define commands here

\theoremstyle{definition}
\newtheorem*{thmplain}{Theorem}

\begin{document}
The second order ordinary differential equation
\begin{align}
\frac{d^2y}{dx^2} \;=\; f\!\left(x,\,y,\,\frac{dy}{dx}\right)
\end{align}
may in certain special cases be solved by using two quadratures, sometimes also by reduction to a \PMlinkname{first order differential equation}{ODE} and a quadrature.\\

If the right hand side of (1) contains at most one of the quantities $x$, $y$ and $\frac{dy}{dx}$, the general solution solution is obtained by two quadratures.\\

\begin{itemize}

\item The equation
\begin{align}
\frac{d^2y}{dx^2} \,=\, f(x)
\end{align}
is considered \PMlinkname{here}{EquationYFx}.\\

\item The equation
\begin{align}
\frac{d^2y}{dx^2} \,=\, f(y)
\end{align}
has as constant solutions all real roots of the equation \,$f(y) = 0$.\, The other solutions can be gotten from the normal system 
\begin{align}
\frac{dy}{dx} \,=\, z, \qquad \frac{dz}{dx} \,=\, f(y)
\end{align}
of (3).\, Dividing the equations (4) we get now\, $\frac{dz}{dy} = \frac{f(y)}{z}$.\, By separation of variables and integration we may write
$$\frac{z^2}{2} = \int\!f(y)\,dy +C_1,$$
whence the first equation of (4) reads
$$\frac{dy}{dx} \,=\, \sqrt{2\!\int\!f(y)\,dy+C_1}.$$
\PMlinkescapetext{Separating} here the variables and integrating give the general integral of (3) in the form
\begin{align}
\int\!\frac{dy}{\sqrt{2\!\int\!f(y)\,dy+C_1}} \;=\; x+C_2.
\end{align}
The \PMlinkname{integration constant}{SolutionsOfOrdinaryDifferentialEquation} $C_1$ has an influence on the form of the integral curves, but $C_2$ only translates them in the direction of the $x$-axis.\\

\item The equation
\begin{align}
\frac{d^2y}{dx^2} \,=\, f(\frac{dy}{dx})
\end{align}
is \PMlinkname{equivalent}{Equivalent3} with the normal system
\begin{align}
\frac{dy}{dx} \,=\, z, \quad \frac{dz}{dx} \,=\, f(z).
\end{align}
If the equation\, $f(z) = 0$\,  has real roots\, $z_1,\,z_2,\,\ldots$,\, these satisfy the latter of the equations (7), and thus, according to the former of them, the differential equation (6) has the solutions\, $y := z_1x+C_1$,\, $y := z_2x+C_2,\;\ldots$.

The other solutions of (6) are obtained by separating the variables and integrating:
\begin{align}
x \,=\, \int\!\frac{dz}{f(z)}+C.
\end{align}
If this antiderivative is expressible in closed form and if then the equation (8) can be solved for $z$, we may write
$$z \,=\, \frac{dy}{dx} \,=\, g(x\!-\!C).$$
Accordingly we have in this case the general solution of the ODE (6):
\begin{align}
y \;=\; \int\!g(x\!-\!C)\,dx+C'.
\end{align}
In other cases, we express also $y$ as a function of $z$.\, By the chain rule, the normal system (7) yields
$$\frac{dy}{dz} \,=\, \frac{dy}{dx}\cdot\frac{dx}{dz} \,=\, \frac{z}{f(z)},$$
whence
$$y = \int\frac{z\,dz}{f(z)}+C'.$$
Thus the general solution of (6) reads now in a parametric form as
\begin{align}
x \,=\, \int\!\frac{dz}{f(z)}+C, \qquad y = \int\frac{z\,dz}{f(z)}+C'.
\end{align}
The equations 10 show that a translation of any integral curve yields another integral curve.

\end{itemize}


%%%%%
%%%%%
\end{document}
