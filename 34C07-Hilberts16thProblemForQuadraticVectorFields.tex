\documentclass[12pt]{article}
\usepackage{pmmeta}
\pmcanonicalname{Hilberts16thProblemForQuadraticVectorFields}
\pmcreated{2013-03-22 14:03:35}
\pmmodified{2013-03-22 14:03:35}
\pmowner{Daume}{40}
\pmmodifier{Daume}{40}
\pmtitle{Hilbert's 16th problem for quadratic vector fields}
\pmrecord{11}{35415}
\pmprivacy{1}
\pmauthor{Daume}{40}
\pmtype{Conjecture}
\pmcomment{trigger rebuild}
\pmclassification{msc}{34C07}

\endmetadata

% this is the default PlanetMath preamble.  as your knowledge
% of TeX increases, you will probably want to edit this, but
% it should be fine as is for beginners.

% almost certainly you want these
\usepackage{amssymb}
\usepackage{amsmath}
\usepackage{amsfonts}

% used for TeXing text within eps files
%\usepackage{psfrag}
% need this for including graphics (\includegraphics)
%\usepackage{graphicx}
% for neatly defining theorems and propositions
%\usepackage{amsthm}
% making logically defined graphics
%%%\usepackage{xypic} 

% there are many more packages, add them here as you need them

% define commands here
\begin{document}
Find a maximum natural number $H(2)$ and relative position of limit cycles of a vector field
\begin{eqnarray*}
\dot{x} = p(x,y) &=&\sum_{i+j=0}^2 a_{ij}x^iy^j \\
\dot{y} = q(x,y) &=& \sum_{i+j=0}^2 b_{ij}x^iy^j
\end{eqnarray*}
\cite{1}.\\
As of now neither part of the problem \textit{(i.e. the bound and the positions of the limit cycles)} are solved.  Although R. Bam\`on in 1986 showed \cite{2} that a quadratic vector field has finite number of limit cycles.  In 1980 Shi Songling \cite{3} and also independently Chen Lan-Sun and Wang Ming-Shu \cite{4} showed an example of a quadratic vector field which has four limit cycles \textit{(i.e. $H(2)\geq 4$)}.\\

\textbf{Example by Shi Songling:}\\
The following system
\begin{eqnarray*}
\dot{x}=& \lambda x - y - 10x^2 + (5+\delta)xy + y^2 \\
\dot{y}=& x + x^2 + (-25 + 8\epsilon - 9\delta)xy
\end{eqnarray*}
has four limit cycles when $0<-\lambda\ll -\epsilon\ll- \delta\ll 1$. \cite{4}\\

\textbf{Example by Chen Lan-sun and Wang Ming-Shu:}\\
The following system
\begin{eqnarray*}
\dot{x}=& -y -\delta_2x - 3x^2 + (1-\delta_1)xy + y^2\\
\dot{y}=& x(1+\frac{2}{3}x - 3y)
\end{eqnarray*}
has four limit cycles when $0<\delta_2\ll\delta_1\ll 1$. \cite{4}

\begin{thebibliography}{1}
\bibitem[DRR]{1} Dumortier, F., Roussarie, R., Rousseau, C.: Hilbert's 16th Problem for Quadratic Vector Fields. Journal of Differential Equations 110, 86-133, 1994.
\bibitem[BR]{2} R. Bam\`on: Quadratic vector fields in the plane have a finite number of limit cycles, Publ. I.H.E.S. 64 (1986), 111-142.
\bibitem[SS]{3} Shi Songling, A concrete example of the existence of four limit cycles for plane quadratic systems, Scientia Sinica 23 (1980), 154-158.
\bibitem[ZTWZ]{4} Zhang Zhi-fen, Ding Tong-ren, Huang Wen-zoa, Dong Zhen-xi.  Qualitative Theory of Differential Equations.  American Mathematical Society, Providence, 1992.
\end{thebibliography}
%%%%%
%%%%%
\end{document}
