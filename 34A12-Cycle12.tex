\documentclass[12pt]{article}
\usepackage{pmmeta}
\pmcanonicalname{Cycle12}
\pmcreated{2013-03-22 15:00:51}
\pmmodified{2013-03-22 15:00:51}
\pmowner{Daume}{40}
\pmmodifier{Daume}{40}
\pmtitle{cycle}
\pmrecord{6}{36721}
\pmprivacy{1}
\pmauthor{Daume}{40}
\pmtype{Definition}
\pmcomment{trigger rebuild}
\pmclassification{msc}{34A12}
\pmclassification{msc}{34C07}
\pmsynonym{periodic solution}{Cycle12}
\pmsynonym{stable periodic solution}{Cycle12}
\pmsynonym{unstable periodic solution}{Cycle12}
\pmsynonym{asymptotically stable periodic solution}{Cycle12}
\pmdefines{period}
\pmdefines{stable cycle}
\pmdefines{unstable cycle}
\pmdefines{asymptotically stable cycle}

% this is the default PlanetMath preamble.  as your knowledge
% of TeX increases, you will probably want to edit this, but
% it should be fine as is for beginners.

% almost certainly you want these
\usepackage{amssymb}
\usepackage{amsmath}
\usepackage{amsfonts}
\usepackage{amsthm}

% used for TeXing text within eps files
%\usepackage{psfrag}
% need this for including graphics (\includegraphics)
%\usepackage{graphicx}
% making logically defined graphics
%%%\usepackage{xypic} 

% there are many more packages, add them here as you need them

% define commands here

% The below lines should work as the command
% \renewcommand{\bibname}{References}
% without creating havoc when rendering an entry in
% the page-image mode.
\makeatletter
\@ifundefined{bibname}{}{\renewcommand{\bibname}{References}}
\makeatother

\newtheorem{thm}{Theorem}
\newtheorem{defn}{Definition}
\newtheorem{prop}{Proposition}
\newtheorem{lemma}{Lemma}
\newtheorem{cor}{Corollary}
\begin{document}
\PMlinkescapeword{closed}
\PMlinkescapeword{example}

Let 
$$\dot{x}=f(x)$$
be an autonomous ordinary differential equation defined by the vector field $f\colon V \to V$ then $x(t)\in V$ a solution of the system is a \emph{cycle}\textit{(or \emph{periodic solution})} if it is a closed solution which is not an equilibrium point.  The \emph{period} of a cycle is the smallest positive $T$ such that $x(t)=x(t+T)$.\\
Let $\phi_t(x)$ be the flow defined by the above ODE and $d$ be the metric of $V$ then:\\
A cycle, $\Gamma$, is a \emph{stable cycle} if for all $\epsilon>0$ there exists a neighborhood $U$ of $\Gamma$ such that for all $x\in U$, $d(\phi_t(x),\Gamma)< \epsilon$.\\
A cycle, $\Gamma$, is \emph{unstable cycle} if it is not a stable cycle.\\
A cycle, $\Gamma$, is \emph{asymptotically stable cycle} if for all $x\in U$ where $U$ is a neighborhood of $\Gamma$, $\lim_{t\to\infty}d(\phi_t(x),\Gamma)=0$.\cite{PL}\\

\textbf{example:}\\
Let
\begin{eqnarray*}
\dot{x} & = & -y\\
\dot{y} & = & x
\end{eqnarray*}
then the above autonomous ordinary differential equations with initial value condition $(x(0),y(0))=(1,0)$ has a solution which is a stable cycle.  Namely the solution defined by
\begin{eqnarray*}
x(t) & = & \cos t\\
y(t) & = & \sin t
\end{eqnarray*}
which has a period of $2\pi$.

\begin{thebibliography}{2}
\bibitem[PL]{PL} Perko, Lawrence: Differential Equations and Dynamical Systems \textit{(Third Edition)}. Springer, New York, 2001.
\end{thebibliography}
%%%%%
%%%%%
\end{document}
