\documentclass[12pt]{article}
\usepackage{pmmeta}
\pmcanonicalname{DAlembertsEquation}
\pmcreated{2013-03-22 14:31:05}
\pmmodified{2013-03-22 14:31:05}
\pmowner{pahio}{2872}
\pmmodifier{pahio}{2872}
\pmtitle{d'Alembert's equation}
\pmrecord{16}{36058}
\pmprivacy{1}
\pmauthor{pahio}{2872}
\pmtype{Derivation}
\pmcomment{trigger rebuild}
\pmclassification{msc}{34A05}
\pmsynonym{Lagrange equation}{DAlembertsEquation}
\pmrelated{ClairautsEquation}
\pmrelated{ContraharmonicProportion}
\pmrelated{DerivativeAsParameterForSolvingDifferentialEquations}

\endmetadata

% this is the default PlanetMath preamble.  as your knowledge
% of TeX increases, you will probably want to edit this, but
% it should be fine as is for beginners.

% almost certainly you want these
\usepackage{amssymb}
\usepackage{amsmath}
\usepackage{amsfonts}

% used for TeXing text within eps files
%\usepackage{psfrag}
% need this for including graphics (\includegraphics)
%\usepackage{graphicx}
% for neatly defining theorems and propositions
%\usepackage{amsthm}
% making logically defined graphics
%%%\usepackage{xypic}

% there are many more packages, add them here as you need them

% define commands here
\begin{document}
The first \PMlinkescapetext{order} differential equation
          $$y = \varphi(\frac{dy}{dx})\cdot x+\psi(\frac{dy}{dx})$$
is called {\em d'Alembert's differential equation}; here $\varphi$ and $\psi$ \PMlinkescapetext{mean} some known differentiable real functions.

If we denote \,$\frac{dy}{dx} := p$, the equation is
                      $$y = \varphi(p)\cdot x+\psi(p).$$
We take $p$ as a new variable and derive the equation with respect to $p$, getting
           $$p-\varphi(p) = [x\varphi'(p)+\psi'(p)]\frac{dp}{dx}.$$
If the equation \,$p-\varphi(p) = 0$\, has the roots \,$p = p_1$, $p_2$, ..., $p_k$, then we have \,$\frac{dp_{\nu}}{dx} = 0$\, for all $\nu$'s, and therefore there are the special solutions 
                $$y = p_{\nu}x+\psi(p_{\nu}) \quad (\nu = 1, 2, ..., k)$$
for the original equation. \,If  \,$\varphi(p) \not\equiv p$, then the derived equation may be written as
 $$\frac{dx}{dp} = \frac{\varphi'(p)}{p-\varphi(p)}x+\frac{\psi'(p)}{p-\varphi(p)},$$
which linear differential equation has the solution \,$x = x(p, C)$\, with the integration constant $C$. \,Thus we get the general solution of d'Alembert's equation as a parametric \PMlinkescapetext{representation}
\[\begin{cases}       
        x = x(p, C),\\
        y = \varphi(p)x(p, C)+\psi(p)
\end{cases}\]
of the integral curves.
%%%%%
%%%%%
\end{document}
