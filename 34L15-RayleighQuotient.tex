\documentclass[12pt]{article}
\usepackage{pmmeta}
\pmcanonicalname{RayleighQuotient}
\pmcreated{2013-03-22 13:39:17}
\pmmodified{2013-03-22 13:39:17}
\pmowner{alozano}{2414}
\pmmodifier{alozano}{2414}
\pmtitle{Rayleigh quotient}
\pmrecord{9}{34308}
\pmprivacy{1}
\pmauthor{alozano}{2414}
\pmtype{Definition}
\pmcomment{trigger rebuild}
\pmclassification{msc}{34L15}
\pmdefines{variational method}

\endmetadata

% this is the default PlanetMath preamble.  as your knowledge
% of TeX increases, you will probably want to edit this, but
% it should be fine as is for beginners.

% almost certainly you want these
\usepackage{amssymb}
\usepackage{amsmath}
\usepackage{amsthm}
\usepackage{amsfonts}
\newcommand{\mv}[1]{\mathbf{#1}}

% used for TeXing text within eps files
%\usepackage{psfrag}
% need this for including graphics (\includegraphics)
%\usepackage{graphicx}
% for neatly defining theorems and propositions
%\usepackage{amsthm}
% making logically defined graphics
%%%\usepackage{xypic}

% there are many more packages, add them here as you need them

% define commands here

\newtheorem*{defn}{Definition}
\begin{document}
\begin{defn} The {\bf Rayleigh quotient}, $R_{\mv{A}}$, to the Hermitian matrix $\mv{A}$ is defined as \\
\begin{displaymath}
R_{\mv{A}}(\mv{x})=\frac{\mv{x}^H \mv{A} \mv{x}}{\mv{x}^H \mv{x}}, \quad \mv{x}\neq \mv{0},
\end{displaymath}where $\mv{x}^H$ is the Hermitian conjugate of $\mv{x}$.
\end{defn}

The importance of this quantity (in fact, the reason Rayleigh first 
introduced it) is that its critical values are the eigenvectors
of $A$ and the values of the quotient at these special vectors are the
corresponding eigenvalues.  This observation leads to the \emph{variational 
method} for computing the spectrum of a positive matrix (either exactly or
approximately).  Namely, one first minimizes the Rayleigh quotient over the
whole vector space.  This gives the lowest eigenvalue and corresponding
eigenvector.  Next, one restricts attention to the orthogonal complement
of the eigenvector found in the first step and minimizes over this subspace.
That produces the next lowest eigenvalue and corresponding eigenvector.  One
can continue this process recursively.  At each step, one minimizes the 
Rayleigh quotient over the subspace orthogonal to all the vectors found in 
the preceding steps to find another eigenvalue and its corresponding 
eigenvector.

This concept of Rayleigh quotient also makes sense in the more general 
setting when $A$ is a Hermitian operator on a Hilbert space.  Furthermore,
it is possible to make use of the Rayleigh-Ritz method in cases where the
operator has a discrete spectrum bounded from below, such as the Laplace
operator on a compact domain.  This method is often employed in practise
because, in physical applications, one is oftentimes interested in only the 
lowest eigenvalue or perhaps the first few lowest eigenvalues and not so
concerned with the rest of the spectrum.
%%%%%
%%%%%
\end{document}
