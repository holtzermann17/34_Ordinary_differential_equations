\documentclass[12pt]{article}
\usepackage{pmmeta}
\pmcanonicalname{Nullcline}
\pmcreated{2013-03-22 14:27:14}
\pmmodified{2013-03-22 14:27:14}
\pmowner{Daume}{40}
\pmmodifier{Daume}{40}
\pmtitle{nullcline}
\pmrecord{5}{35972}
\pmprivacy{1}
\pmauthor{Daume}{40}
\pmtype{Definition}
\pmcomment{trigger rebuild}
\pmclassification{msc}{34C99}

\endmetadata

% this is the default PlanetMath preamble.  as your knowledge
% of TeX increases, you will probably want to edit this, but
% it should be fine as is for beginners.

% almost certainly you want these
\usepackage{amssymb}
\usepackage{amsmath}
\usepackage{amsfonts}

% used for TeXing text within eps files
%\usepackage{psfrag}
% need this for including graphics (\includegraphics)
%\usepackage{graphicx}
% for neatly defining theorems and propositions
%\usepackage{amsthm}
% making logically defined graphics
%%%\usepackage{xypic} 

% there are many more packages, add them here as you need them

% define commands here

% The below lines should work as the command
% \renewcommand{\bibname}{References}
% without creating havoc when rendering an entry in
% the page-image mode.
\makeatletter
\@ifundefined{bibname}{}{\renewcommand{\bibname}{References}}
\makeatother
\begin{document}
Let
\begin{eqnarray*}
\dot{x}_1 &=& f_1(x_1,\ldots,x_n)\\
& \vdots & \\
\dot{x}_n &=& f_n(x_1,\ldots,x_n)
\end{eqnarray*}
be a system of first order ordinary differential equation.  The $x_j$ \emph{nullcline} is the set of points which satisfy $f_j(x_1,\ldots,x_n)=0$.  Note that at an intersection point of all the nullclines implies that 
\begin{eqnarray*}
0 &=& f_1(x_1,\ldots,x_n)\\
& \vdots & \\
0 &=& f_n(x_1,\ldots,x_n).
\end{eqnarray*}
Hence the intersection point of all the nullclines is an equilibrium point of the system.
\\\\
\textbf{example:}
\begin{itemize}
\item see some qualitative analysis of FitzHugh-Nagumo equation using nullclines
\end{itemize}
%%%%%
%%%%%
\end{document}
