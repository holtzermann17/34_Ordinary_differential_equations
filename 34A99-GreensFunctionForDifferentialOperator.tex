\documentclass[12pt]{article}
\usepackage{pmmeta}
\pmcanonicalname{GreensFunctionForDifferentialOperator}
\pmcreated{2013-03-22 14:43:39}
\pmmodified{2013-03-22 14:43:39}
\pmowner{mathforever}{4370}
\pmmodifier{mathforever}{4370}
\pmtitle{Green's function for differential operator}
\pmrecord{7}{36356}
\pmprivacy{1}
\pmauthor{mathforever}{4370}
\pmtype{Example}
\pmcomment{trigger rebuild}
\pmclassification{msc}{34A99}
\pmclassification{msc}{34A30}

% this is the default PlanetMath preamble.  as your knowledge
% of TeX increases, you will probably want to edit this, but
% it should be fine as is for beginners.

% almost certainly you want these
\usepackage{amssymb}
\usepackage{amsmath}
\usepackage{amsfonts,euscript}

% used for TeXing text within eps files
%\usepackage{psfrag}
% need this for including graphics (\includegraphics)
\usepackage{graphicx}
% for neatly defining theorems and propositions
%\usepackage{amsthm}
% making logically defined graphics
%%%\usepackage{xypic}

% there are many more packages, add them here as you need them

% define commands here
\begin{document}
Assume we are given $g\in\mathcal{C}^0([0,T])$ and we want to find $f\in\mathcal{C}^1([0,T])$ such that
\begin{equation}\label{DiffEq}
    \left\{
    \begin{array}{rcl}
        f'(t) & = & g(t) \\
        f(0)  & = & 0
    \end{array}
    \right.
\end{equation}
Expression (\ref{DiffEq}) is an example of initial value problem for an ordinary differential equation. Let us
show, that (\ref{DiffEq}) can be put into the framework of the definition for Green's function.
\begin{enumerate}
\item $\Omega_x=\Omega_y=[0,T]$.

\item $\EuScript{F}(\Omega_x)=\{ f\in\mathcal{C}^1([0,T])\,|\,f(0)=0 \}$\\
      $\EuScript{G}(\Omega_y)=\mathcal{C}^0([0,T])$.

\item $Af=f'$

\end{enumerate}
Thus (\ref{DiffEq}) can be written as an operator equation
\begin{equation}\label{OpEq}
    Af=g.
\end{equation}
To find the Green's function for (\ref{OpEq}) we proceed as follows:
$$
    f(t)=\delta_t(A^{-1}g)=\int\limits_0^t g(t')\,dt'=\int\limits_0^T G(t,t')g(t')\,dt',
$$
where $G(t,t')$ has the following form:
\begin{equation}\label{GrFn}
    G(t,t')=\left\{
                \begin{array}{rl}
                    1, & 0\leq t \leq t'\\
                    0, & t'<   t \leq T
                \end{array}
            \right.
\end{equation}
Thus, function (\ref{GrFn}) is the Green's function for the operator equation (\ref{OpEq}) and
then for the problem (\ref{DiffEq}). 

Its graph is presented in Figure~\ref{GrFnPic}.

\begin{figure}[h]
\begin{center}
    \includegraphics[scale=.5]{GrFn.eps}
\end{center}
\caption{The Green's function for the problem (\ref{DiffEq}).}
\label{GrFnPic}
\end{figure}
%%%%%
%%%%%
\end{document}
