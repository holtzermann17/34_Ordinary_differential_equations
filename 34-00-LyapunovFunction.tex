\documentclass[12pt]{article}
\usepackage{pmmeta}
\pmcanonicalname{LyapunovFunction}
\pmcreated{2013-03-22 13:42:29}
\pmmodified{2013-03-22 13:42:29}
\pmowner{CWoo}{3771}
\pmmodifier{CWoo}{3771}
\pmtitle{Lyapunov function}
\pmrecord{7}{34386}
\pmprivacy{1}
\pmauthor{CWoo}{3771}
\pmtype{Definition}
\pmcomment{trigger rebuild}
\pmclassification{msc}{34-00}
\pmsynonym{Liapunov function}{LyapunovFunction}

\endmetadata

% this is the default PlanetMath preamble.  as your knowledge
% of TeX increases, you will probably want to edit this, but
% it should be fine as is for beginners.

% almost certainly you want these
\usepackage{amssymb}
\usepackage{amsmath}
\usepackage{amsfonts}

% used for TeXing text within eps files
%\usepackage{psfrag}
% need this for including graphics (\includegraphics)
%\usepackage{graphicx}
% for neatly defining theorems and propositions
%\usepackage{amsthm}
% making logically defined graphics
%%%\usepackage{xypic}

% there are many more packages, add them here as you need them

% define commands here
\begin{document}
Suppose we are given an autonomous system of first order
differential equations.

\[
\frac{dx}{dt}=F(x,y)\quad\frac{dy}{dt}=G(x,y)
\]


Let the origin be an isolated critical point of the above system.

A function $V(x,y)$ that is of class $C^{1}$ and satisfies
$V(0,0)=0$ is called a \emph{Lyapunov function} if every open ball
$B_{\delta}(0,0)$ contains at least one point where $V>0.$ If
there happens to exist $\delta^{*}$ such that the function
$\dot{V}$, given by

\[
\dot{V}(x,y)=V_{x}(x,y)F(x,y)+V_{y}(x,y)G(x,y)
\]

is positive definite in $B_{\delta}^{*}(0,0)$.  Then the origin is
an unstable critical point of the system.
%%%%%
%%%%%
\end{document}
