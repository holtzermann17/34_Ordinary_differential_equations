\documentclass[12pt]{article}
\usepackage{pmmeta}
\pmcanonicalname{SeparatingSolution}
\pmcreated{2013-03-22 14:06:17}
\pmmodified{2013-03-22 14:06:17}
\pmowner{Daume}{40}
\pmmodifier{Daume}{40}
\pmtitle{separating solution}
\pmrecord{10}{35505}
\pmprivacy{1}
\pmauthor{Daume}{40}
\pmtype{Definition}
\pmcomment{trigger rebuild}
\pmclassification{msc}{34A99}
\pmclassification{msc}{34-00}

% this is the default PlanetMath preamble.  as your knowledge
% of TeX increases, you will probably want to edit this, but
% it should be fine as is for beginners.

% almost certainly you want these
\usepackage{amssymb}
\usepackage{amsmath}
\usepackage{amsfonts}

% used for TeXing text within eps files
%\usepackage{psfrag}
% need this for including graphics (\includegraphics)
%\usepackage{graphicx}
% for neatly defining theorems and propositions
%\usepackage{amsthm}
% making logically defined graphics
%%%\usepackage{xypic} 

% there are many more packages, add them here as you need them

% define commands here
\begin{document}
\PMlinkescapeword{planar}

A \emph{separating solution} of a smooth planar dynamical system is an oriented smooth curve such that
\begin{itemize}
\item the curve is composed of trajectories of the dynamical system,
\item the curve does not go through any equilibrium point,
\item the curve is the boundary of some region in the plane.
\end{itemize}
The orientation of the curve is the same as the orientation of the solutions.  In addition the curve does not need to be connected.
\cite{1}
\begin{thebibliography}{1}
\bibitem[KGA]{1} Khovanski\u\i, A.G.: Fewnomials. American Mathematical Society, Providence, 1991.
\end{thebibliography}
%%%%%
%%%%%
\end{document}
