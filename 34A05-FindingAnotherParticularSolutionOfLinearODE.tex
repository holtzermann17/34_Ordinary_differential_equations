\documentclass[12pt]{article}
\usepackage{pmmeta}
\pmcanonicalname{FindingAnotherParticularSolutionOfLinearODE}
\pmcreated{2014-02-28 14:31:42}
\pmmodified{2014-02-28 14:31:42}
\pmowner{pahio}{2872}
\pmmodifier{pahio}{2872}
\pmtitle{finding another particular solution of linear ODE}
\pmrecord{10}{88035}
\pmprivacy{1}
\pmauthor{pahio}{2872}
\pmtype{Algorithm}
\pmclassification{msc}{34A05}

% this is the default PlanetMath preamble.  as your knowledge
% of TeX increases, you will probably want to edit this, but
% it should be fine as is for beginners.

% almost certainly you want these
\usepackage{amssymb}
\usepackage{amsmath}
\usepackage{amsfonts}

% need this for including graphics (\includegraphics)
\usepackage{graphicx}
% for neatly defining theorems and propositions
\usepackage{amsthm}

% making logically defined graphics
%\usepackage{xypic}
% used for TeXing text within eps files
%\usepackage{psfrag}

% there are many more packages, add them here as you need them

% define commands here

\begin{document}
Consider the 
\PMlinkname{homogeneous}{HomogeneousLinearDifferentialEquation} 
second-order linear ordinary differential equation
\begin{align}
      y''\!+\!P(x)y'\!+\!Q(x)y \;=\; 0.
\end{align}

If one knows one 
\PMlinkname{particular solution}{SolutionsOfOrdinaryDifferentialEquation}\, 
$y = y_1(x) \not\equiv 0$\, of (1), it's possible to derive 
from it via two quadratures another solution\, $y_2(x)$,\, 
linearly independent on\, $y_1(x)$;\, thus one can write the 
general solution
     $$y \;=\; C_1y_1(x)\!+\!C_2y_2(x)$$
of that homogeneous differential equation.\\

We will now show the derivation procedure.

We put
\begin{align}
     y \;=\; uv
\end{align}
which renders (1) to
\begin{align}
(v''+Pv'+Qv)u+(2v'+Pv)u'+u''v \;=\; 0.
\end{align}
Here one can choose\, $v := y_1(x)$, whence the first addend 
vanishes, and (3) gets the form
\begin{align}
(2y_1'+Py_1)u'+y_1u'' \;=\;0.
\end{align}
This equation may be written as\, $\frac{u''}{u'} = 
-2\frac{y_1'}{y_1}-P$, which is integrated to
     $$\ln\left|\frac{du}{dx}\right| 
     \;=\; \ln\frac{1}{y_1^2}-\int P\,dx+\mbox{constant},$$
i.e.
  $$\frac{du}{dx} \;=\; \frac{C}{y_1^2}e^{-\int P\,dx}.$$
A new integration results from this the general solution of (4):
   $$u \;=\; C\int\frac{e^{-\int P\,dx}}{y_1^2}\,dx+C'.$$
Thus by (2), we have obtained the wanted other solution
$$y_2(x) \;=\; y_1(x)\int\frac{e^{-\int P\,dx}}{y_1^2}\,dx$$
which is clearly linearly independent on y_1(x).\\

Consequently, we can express the general solution of the 
differential equation (1) as
  $$y \;=\; y_1(x)u 
  \;=\; C_1y_1(x)+C_2y_1(x)\int\frac{e^{-\int P\,dx}}{y_1^2}\,dx,$$
where $C_1$ and $C_2$ are arbitrary constants.\\

\textbf{Remark.}\, The substitution 
$$y \;:=\; e^{-\frac{1}{2}\int P(x)\,dx}u$$
converts the equation (1) into the form
$$\frac{d^2u}{dx^2}+(Q-\frac{P^2}{4}-\frac{P'}{2})u \;=\;0$$
not containing the derivative $\frac{du}{dx}$.


\begin{thebibliography}{8}
\bibitem{lindelof}{\sc Ernst Lindel\"of}: {\em Differentiali- ja integralilasku
ja sen sovellutukset III.1}.\, Mercatorin Kirjapaino Osakeyhti\"o, Helsinki (1935).
\end{thebibliography}



\end{document}
