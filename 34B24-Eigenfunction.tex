\documentclass[12pt]{article}
\usepackage{pmmeta}
\pmcanonicalname{Eigenfunction}
\pmcreated{2013-03-22 12:48:00}
\pmmodified{2013-03-22 12:48:00}
\pmowner{tensorking}{373}
\pmmodifier{tensorking}{373}
\pmtitle{eigenfunction}
\pmrecord{8}{33117}
\pmprivacy{1}
\pmauthor{tensorking}{373}
\pmtype{Definition}
\pmcomment{trigger rebuild}
\pmclassification{msc}{34B24}
\pmsynonym{characteristics function}{Eigenfunction}
%\pmkeywords{Sturm-Liouville}
\pmdefines{solution of system}

\endmetadata

% this is the default PlanetMath preamble.  as your knowledge
% of TeX increases, you will probably want to edit this, but
% it should be fine as is for beginners.

% almost certainly you want these
\usepackage{amssymb}
\usepackage{amsmath}
\usepackage{amsfonts}

% used for TeXing text within eps files
%\usepackage{psfrag}
% need this for including graphics (\includegraphics)
%\usepackage{graphicx}
% for neatly defining theorems and propositions
%\usepackage{amsthm}
% making logically defined graphics
%%%\usepackage{xypic}

% there are many more packages, add them here as you need them

% define commands here
\begin{document}
Consider the Sturm-Liouville system given by

\begin{equation}
\frac{d}{dx}\left[p(x)\frac{dy}{dx}\right]+q(x)y+\lambda
r(x)y=0\;\;\;\;\;\;a\leq x\leq b \label{stuff}
\end{equation}

\begin{equation}
a_{1}y(a)+a_{2}y^{\prime}(a)=0,\;\;\;
\;\;\;b_{1}y(b)+b_{2}y^{\prime}(b)=0, \label{stuff1}
\end{equation}

where $a_{i},b_{i}\in \mathbb{R}$ with $i\in \{1,2\}$ and
$p(x),q(x),r(x)$ are differentiable functions and
$\lambda\in\mathbb{R}$. A non zero solution of the system defined
by \eqref{stuff} and \eqref{stuff1} exists in general for a
specified $\lambda$.  The functions corresponding to that
specified $\lambda$ are called eigenfunctions.



More generally, if $D$ is some linear differential operator, and
$\lambda\in \mathbb{R}$ and $f$ is a function such that
$Df=\lambda f$ then we say $f$ is an eigenfunction of $D$ with
eigenvalue $\lambda$.
%%%%%
%%%%%
\end{document}
