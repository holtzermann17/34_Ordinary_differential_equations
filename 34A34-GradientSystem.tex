\documentclass[12pt]{article}
\usepackage{pmmeta}
\pmcanonicalname{GradientSystem}
\pmcreated{2013-03-22 15:14:25}
\pmmodified{2013-03-22 15:14:25}
\pmowner{Daume}{40}
\pmmodifier{Daume}{40}
\pmtitle{gradient system}
\pmrecord{7}{37014}
\pmprivacy{1}
\pmauthor{Daume}{40}
\pmtype{Definition}
\pmcomment{trigger rebuild}
\pmclassification{msc}{34A34}

% this is the default PlanetMath preamble.  as your knowledge
% of TeX increases, you will probably want to edit this, but
% it should be fine as is for beginners.

% almost certainly you want these
\usepackage{amssymb}
\usepackage{amsmath}
\usepackage{amsfonts}
\usepackage{amsthm}

% used for TeXing text within eps files
%\usepackage{psfrag}
% need this for including graphics (\includegraphics)
%\usepackage{graphicx}
% making logically defined graphics
%%%\usepackage{xypic} 

% there are many more packages, add them here as you need them

% define commands here

% The below lines should work as the command
% \renewcommand{\bibname}{References}
% without creating havoc when rendering an entry in
% the page-image mode.
\makeatletter
\@ifundefined{bibname}{}{\renewcommand{\bibname}{References}}
\makeatother

\newtheorem{thm}{Theorem}
\newtheorem{defn}{Definition}
\newtheorem{prop}{Proposition}
\newtheorem{lemma}{Lemma}
\newtheorem{cor}{Corollary}
\begin{document}
\PMlinkescapeword{properties}

A \emph{gradient system} in $\mathbb{R}^n$ is an autonomous ordinary differential equation
\begin{equation}
\dot{x}=-\operatorname{grad}V(x)\label{eq}
\end{equation}
defined by the gradient of $V$ where $V:\mathbb{R}^n\to \mathbb{R}$ and $V\in C^\infty$.  The following results can be deduced from the definition of a gradient system.\\
\textbf{Properties:}
\begin{itemize}
\item The eigenvalues of the linearization of (\ref{eq}) evaluated at equilibrium point are real.
\item If $x_0$ is an isolated minimum of $V$ then $x_0$ is an asymptotically stable solution of (\ref{eq})
\item If $x(t)$ is a solution of (\ref{eq}) that is not an equilibrium point then $V(x(t))$ is a strictly decreasing function and is perpendicular to the level curves of $V$.
\item There does not exists periodic solutions of (\ref{eq}).
\end{itemize}

\begin{thebibliography}{1}
\bibitem[HSD]{HSD} Hirsch, W. Morris, Smale, Stephen, Devaney, L. Robert: Differential Equations, Dynamical Systems \& An Introduction to Chaos. Elsevier Academic Press, New York, 2004.
\end{thebibliography}
%%%%%
%%%%%
\end{document}
