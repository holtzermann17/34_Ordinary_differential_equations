\documentclass[12pt]{article}
\usepackage{pmmeta}
\pmcanonicalname{BautinsTheorem}
\pmcreated{2013-03-22 14:28:46}
\pmmodified{2013-03-22 14:28:46}
\pmowner{Daume}{40}
\pmmodifier{Daume}{40}
\pmtitle{Bautin's theorem}
\pmrecord{5}{36011}
\pmprivacy{1}
\pmauthor{Daume}{40}
\pmtype{Theorem}
\pmcomment{trigger rebuild}
\pmclassification{msc}{34C07}

% this is the default PlanetMath preamble.  as your knowledge
% of TeX increases, you will probably want to edit this, but
% it should be fine as is for beginners.

% almost certainly you want these
\usepackage{amssymb}
\usepackage{amsmath}
\usepackage{amsfonts}

% used for TeXing text within eps files
%\usepackage{psfrag}
% need this for including graphics (\includegraphics)
%\usepackage{graphicx}
% for neatly defining theorems and propositions
%\usepackage{amsthm}
% making logically defined graphics
%%%\usepackage{xypic} 

% there are many more packages, add them here as you need them

% define commands here

% The below lines should work as the command
% \renewcommand{\bibname}{References}
% without creating havoc when rendering an entry in
% the page-image mode.
\makeatletter
\@ifundefined{bibname}{}{\renewcommand{\bibname}{References}}
\makeatother
\begin{document}
\PMlinkescapeword{type}
\PMlinkescapeword{focus}

There are at most three limit cycles which can appear in the following quadratic system
\begin{eqnarray*}
\dot{x} = p(x,y) &=&\sum_{i+j=0}^2 a_{ij}x^iy^j \\
\dot{y} = q(x,y) &=& \sum_{i+j=0}^2 b_{ij}x^iy^j
\end{eqnarray*}
from a singular point, if its type is either a focus or a center.  \\

In 1939 N.N. Bautin claimed the above result and in 1952 submitted the proof \cite{BNN1}. \cite{GAV}
\begin{thebibliography}{1}
\bibitem[GAV]{GAV} Gaiko, A., Valery: Global Bifurcation Theory and Hilbert's Sixteenth Problem. Kluwer Academic Publishers, London, 2003.
\bibitem[BNN1]{BNN1} Bautin, N.N.: On the number of limit cycles appearing from an equilibrium point of the focus or center type under varying coefficients. Matem. SB., 30:181-196, 1952. (written in Russian)
\bibitem[BNN2]{BNN2} Bautin, N.N.: On the number of limit cycles appearing from an equilibrium point of the focus or center type under varying coefficients. Translation of the American Mathematical Society, 100, 1954. 
\end{thebibliography}
%%%%%
%%%%%
\end{document}
