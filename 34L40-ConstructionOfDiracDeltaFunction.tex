\documentclass[12pt]{article}
\usepackage{pmmeta}
\pmcanonicalname{ConstructionOfDiracDeltaFunction}
\pmcreated{2013-03-22 12:35:48}
\pmmodified{2013-03-22 12:35:48}
\pmowner{djao}{24}
\pmmodifier{djao}{24}
\pmtitle{construction of Dirac delta function}
\pmrecord{5}{32848}
\pmprivacy{1}
\pmauthor{djao}{24}
\pmtype{Derivation}
\pmcomment{trigger rebuild}
\pmclassification{msc}{34L40}
\pmclassification{msc}{26E35}

\endmetadata

% this is the default PlanetMath preamble.  as your knowledge
% of TeX increases, you will probably want to edit this, but
% it should be fine as is for beginners.

% almost certainly you want these
\usepackage{amssymb}
\usepackage{amsmath}
\usepackage{amsfonts}

% used for TeXing text within eps files
%\usepackage{psfrag}
% need this for including graphics (\includegraphics)
%\usepackage{graphicx}
% for neatly defining theorems and propositions
%\usepackage{amsthm}
% making logically defined graphics
%%%\usepackage{xypic} 

% there are many more packages, add them here as you need them

% define commands here
\newcommand{\R}{\mathbb{R}}
\begin{document}
The Dirac delta function is notorious in mathematical circles for having no actual \PMlinkescapetext{realization} as a function. However, a little known secret is that in the domain of nonstandard analysis, the Dirac delta function admits a completely legitimate construction as an actual function. We give this construction here.

Choose any positive infinitesimal $\varepsilon$ and define the hyperreal valued function $\delta:\,^*\R \longrightarrow\,^*\R$ by
$$
\delta(x) :=
\begin{cases}
1/\varepsilon & -\varepsilon/2 < x < \varepsilon/2, \\
0 & \text{otherwise.}
\end{cases}
$$
We verify that the above function satisfies the required properties of the Dirac delta function. By definition, $\delta(x) = 0$ for all nonzero real numbers $x$. Moreover,
$$
\int_{-\infty}^\infty \delta(x)\ dx = \int_{-\varepsilon/2}^{\varepsilon/2} \frac{1}{\varepsilon} \ dx = 1,
$$
so the integral property is satisfied. Finally, for any {\em continuous} real function $f: \R \longrightarrow \R$, choose an infinitesimal $z > 0$ such that $|f(x) - f(0)| < z$ for all $|x| < \varepsilon/2$; then
$$
\varepsilon \cdot \frac{f(0) - z}{\varepsilon} < \int_{-\infty}^\infty \delta(x) f(x)\ dx < \varepsilon \cdot \frac{f(0) + z}{\varepsilon}
$$
which implies that $\int_{-\infty}^\infty \delta(x) f(x)\ dx$ is within an infinitesimal of $f(0)$, and thus has real part equal to $f(0)$.
%%%%%
%%%%%
\end{document}
